\documentclass[a4paper,10pt]{article}
\usepackage[utf8]{inputenc}
\usepackage{verbatim}
\usepackage{amssymb}
\usepackage{amsmath}
\usepackage{url}
\usepackage{hyperref}
\usepackage{xcolor}
\usepackage{algorithm}
\usepackage{algpseudocode}
\usepackage{algorithmicx}
\usepackage{graphicx}

%opening
\title{Notes for libEMMI\_MGFD}
\author{Pengliang Yang\\
Harbin Institute of Technology, China\\
E-mail: ypl.2100@gmail.com}

\begin{document}

\maketitle


\section{Input parameters}
\begin{itemize}
\item \verb|mode|: \texttt{mode=0}, forward modelling; \texttt{mode=1}, 3D CSEM inversion;
\item \verb|freqs:| the frequencies used for CSEM modelling and inversion, a number of frequencies can be given by comma separated values;
\item \verb|chsrc:| source channels (i.e. Ex, Ey, Ez, Hx, Hy, Hz)
\item \verb|chrec:| receiver channels (i.e. Ex, Ey, Ez, Hx, Hy, Hz)
\item \verb|nx,ny,nz|: number of intervals in x, y and z axes for input resistivity model on equispaced FD grid;
\item \verb|dx,dy,dz|: grid spacing of input resistivity model on equispaced FD grid;
\item \verb|ox,oy,oz|: origin of the 3D coordinates in x, y and z directions;
\item \verb|fbathy|: a binary input file of size \texttt{nx*ny} to specify bathymetry information;
\item \verb|frho11,frho22,frho33|: binary file of size \texttt{nx*ny*z} to specify resistivities;
\item \verb|fsrc|: an ASCII file to specify source locations and orientations;
\item \verb|frec|: an ASCII file to specify receiver locations and orientations;
\item \verb|fsrcrec|: an ASCII file to specify the connection between source and receivers;
\item \verb|niter|: number of iterations for nonlinear optimization;
\item \verb|npar|: number of parameters used for inversion, default value=2;
\item \verb|bound|: \texttt{bound=1} uses bounded LBFGS; \texttt{bound=0} does not apply bound constraint;
\item \verb|idxpar|: index of the inversion parameter, default value=1,2 indicating horizontal and vertical resistivities;
\item \verb|minpar|: the minimum values for the physical parameters;
\item \verb|maxpar|: the maximum values for the physical parameters;
\item \verb|gamma1|: strength of 1st order Tikhonov regularization;
\item \verb|gamma2|: strength of Total Variational (TV) regularization;
\end{itemize}
An example job script \verb|run.sh| using the above parameters is listed in the following.

\begin{verbatim}
#!/bin/bash

export OMP_NUM_THREADS=2
mpirun -n 25 ../bin/main mode=1 \
       freqs=0.25,1,2.75 \
       chsrc=Ex \
       chrec=Ex ,Ey,Hx,Hy \
       nx=100 \
       ny=100 \
       nz=100 \
       dx=200 \
       dy=200 \
       dz=40 \
       ox=-10000 \
       oy=-10000 \
       oz=0 \
       fbathy=fbathy \
       frho11=frho_init \
       frho22=frho_init \
       frho33=frho_init \
       fsrc=sources.txt \
       frec=receivers.txt \
       fsrcrec=src_rec_table.txt \
       niter=30 \
       npar=2 \
       bound=1 \
       idxpar=1,2 \
       minpar=1.0,1.0 \
       maxpar=100.0,100.0 \
       gamma1=100 \
       gamma2=0
\end{verbatim}

\section{Source-receiver configuration}
The locations and orientations for every source/transmitter and receiver are written in a 6-column table. The following is an example of source table \texttt{sources.txt} where \texttt{x,y,z} are coordinates, \texttt{azimuth,dip} are orientations, \texttt{iTx} is the index of the transmitter.
\begin{verbatim}
     x             y            z         azimuth       dip   iTx
-2196.15234   -8196.15234    903.652222   30.0000000     0      1
 401.923828   -6696.15234    870.258484   30.0000000     0      2
 3000.00000   -5196.15234    834.029785   30.0000000     0      3
 5598.07617   -3696.15234    810.434204   30.0000000     0      4
 8196.15234   -2196.15234    809.226013   30.0000000     0      5
-3696.15234   -5598.07617    865.961426   30.0000000     0      6
-1098.07617   -4098.07617    832.172302   30.0000000     0      7
 1500.00000   -2598.07617    807.180298   30.0000000     0      8
 4098.07617   -1098.07617    802.881104   30.0000000     0      9
 6696.15234    401.923828    821.753967   30.0000000     0     10
 ......
\end{verbatim}

The following is an example of receiver table \texttt{receivers.txt} where \texttt{x,y,z} are coordinates, \texttt{azimuth,dip} are orientations, \texttt{iRx} is the index of the receiver.
\begin{verbatim}
     x             y              z       azimuth  dip  iRx
-10000.0000     0.00000000     1000.00000     0     0     1
-9800.00000     0.00000000     1000.00000     0     0     2
-9600.00000     0.00000000     1000.00000     0     0     3
-9400.00000     0.00000000     1000.00000     0     0     4
-9200.00000     0.00000000     1000.00000     0     0     5
-9000.00000     0.00000000     1000.00000     0     0     6
-8800.00000     0.00000000     1000.00000     0     0     7
-8600.00000     0.00000000     1000.00000     0     0     8
-8400.00000     0.00000000     1000.00000     0     0     9
-8200.00000     0.00000000     1000.00000     0     0    10
-8000.00000     0.00000000     1000.00000     0     0    11
-7800.00000     0.00000000     1000.00000     0     0    12
-7600.00000     0.00000000     1000.00000     0     0    13
-7400.00000     0.00000000     1000.00000     0     0    14
-7200.00000     0.00000000     1000.00000     0     0    15
-7000.00000     0.00000000     1000.00000     0     0    16
......
\end{verbatim}

The connections between sources and receivers (which receivers record data from which source) must be specified by a source-receiver connection table
\verb|src_rec_table.txt| according to the index of the source and receivers.
\begin{verbatim}
 isrc        irec
   1           1
   1           2
   1           3
   1           4
   1           5
   1           6
   1           7
   1           8
   1           9
   1          10
   ...
   2           1
   2           2
   2           3
   2           4
   2           5
   2           6
   2           7
   2           8
   2           9
   2          10
   ...
\end{verbatim}


\section{Output EMF files}

The simulated CSEM data are stored in ASCII files. An example EM data file \verb|emf_0001.txt| from source index \texttt{0001} includes index of source and receivers (\texttt{iTx}, \texttt{iRx}), the recording channels of the receiver \texttt{chrec}, frequencies in Hz, real and imaginary part of the frequency domain data. They form a 6-column table in the following.
\begin{verbatim}
iTx  iRx   chrec frequency/Hz    Real{E/H}     Imag{E/H}
1     1      Ex     0.25       -8.794095e-15 2.336085e-14
1     2      Ex     0.25       -8.236141e-15 2.632050e-14
1     3      Ex     0.25       -7.451898e-15 2.954083e-14
1     4      Ex     0.25       -6.390576e-15 3.306965e-14
1     5      Ex     0.25       -4.998464e-15 3.694922e-14
1     6      Ex     0.25       -3.228805e-15 4.118009e-14
1     7      Ex     0.25       -1.026235e-15 4.577336e-14
1     8      Ex     0.25       1.681961e-15 5.075206e-14
1     9      Ex     0.25       4.974577e-15 5.613805e-14
1     10     Ex     0.25       8.933074e-15 6.194364e-14
......
1     1      Ex      1   	 -1.045790e-12 	 9.924710e-13
1     2      Ex      1   	 -1.119678e-12 	 1.968149e-12
1     3      Ex      1   	 -8.904390e-13 	 3.433207e-12
1     4      Ex      1   	 -1.103422e-13 	 6.096019e-12
1     5      Ex      1   	 1.431065e-12 	 1.066018e-11
1     6      Ex      1   	 6.458673e-12 	 2.313294e-11
1     7      Ex      1   	 2.208377e-11 	 5.329251e-11
1     8      Ex      1   	 1.162063e-10 	 1.498437e-10
1     9      Ex      1   	 3.530520e-10 	 3.849713e-10
1     10     Ex      1   	 -1.178124e-09 	 6.435691e-10
......
\end{verbatim}


\section{Convergence information on CSEM inversion}

After the 3D inversion, the convergence history of the nonlinear optimization will be stored in an ASCII file named \verb|iterate.txt|.
\begin{verbatim}
 ==========================================================
l-BFGS memory length: 5
Maximum number of iterations: 30
Convergence tolerance: 1.00e-06
maximum number of line search: 5
initial step length: alpha=1
==========================================================
iter    fk       fk/f0      ||gk||    alpha    nls   ngrad
  0   1.08e+03  1.00e+00   5.30e+00  1.00e+00    0     0
  1   9.16e+02  8.49e-01   4.50e+00  4.00e+00    2     3
  2   6.48e+02  6.01e-01   5.15e+00  2.50e-01    2     6
  3   5.96e+02  5.52e-01   8.72e+00  1.00e+00    0     7
  4   4.35e+02  4.03e-01   4.40e+00  1.00e+00    0     8
  5   3.35e+02  3.11e-01   3.20e+00  1.00e+00    0     9
  6   2.58e+02  2.39e-01   5.29e+00  5.00e-01    1    11
  7   2.09e+02  1.94e-01   3.69e+00  1.00e+00    0    12
  8   1.71e+02  1.58e-01   2.00e+00  1.00e+00    0    13
  9   1.44e+02  1.34e-01   1.53e+00  1.00e+00    0    14
 10   1.23e+02  1.14e-01   2.09e+00  1.00e+00    0    15
 11   1.16e+02  1.07e-01   2.28e+00  1.00e+00    0    16
 12   8.96e+01  8.31e-02   2.02e+00  1.00e+00    0    17
 13   7.71e+01  7.15e-02   1.77e+00  1.00e+00    0    18
 14   6.27e+01  5.82e-02   8.97e-01  1.00e+00    0    19
 15   5.48e+01  5.08e-02   1.11e+00  1.00e+00    0    20
 16   4.93e+01  4.57e-02   7.89e-01  1.00e+00    0    21
 17   4.45e+01  4.13e-02   6.47e-01  1.00e+00    0    22
 18   4.06e+01  3.76e-02   7.92e-01  1.00e+00    0    23
 19   3.69e+01  3.42e-02   6.41e-01  1.00e+00    0    24
 20   3.42e+01  3.17e-02   7.62e-01  1.00e+00    0    25
 21   3.30e+01  3.06e-02   9.15e-01  1.00e+00    0    26
 22   3.01e+01  2.79e-02   6.26e-01  1.00e+00    0    27
 23   2.80e+01  2.60e-02   6.31e-01  1.00e+00    0    28
 24   2.66e+01  2.47e-02   6.19e-01  1.00e+00    0    29
 25   2.60e+01  2.41e-02   7.20e-01  1.00e+00    0    30
 26   2.58e+01  2.40e-02   1.13e+00  1.00e+00    0    31
 27   2.50e+01  2.31e-02   8.96e-01  2.00e+00    1    38
 28   2.44e+01  2.26e-02   8.16e-01  1.00e+00    0    39
 29   2.35e+01  2.18e-02   4.30e-01  1.00e+00    0    40
==>Maximum iteration number reached!
\end{verbatim}
In the above example, each columns has clear meaning:
\begin{itemize}
\item \texttt{iter}: the iteration index k;
\item \texttt{fk}: the misfit at the k-th iteration;
\item \texttt{fk/f0}: the normalized misfit at the k-th iteration;
\item \texttt{||gk||}: the norm of the gradient;
\item \texttt{alpha}: step length used in line search;
\item \texttt{nls}: number of line search at the k-th iteration;
\item \texttt{ngrad}: number of gradient evaluations
\end{itemize}

\section{The Green's function and the reciprocity}\label{appendix:reciprocity}

Assume only electrical current $J_j(x_s,\omega)=\delta(x-x_s)e_j$ where $e_j$ is the $j$-directed unit vector. We have
\begin{equation}\label{eq:E|EH|E}
  \begin{cases}
    \nabla \times G_{ij}^{E|E} -\mathrm{i}\omega\mu  G_{ij}^{H|E} &= 0 \\
    \nabla \times G_{ij}^{H|E} -\sigma  G_{ij}^{E|E}  &=\delta(x-x_s)e_j
  \end{cases},
\end{equation}
which defines two Green's function $G_{ij}^{E|E}$ and $G_{ij}^{H|E}$: 
$G_{ij}^{E|E}$ is the $i$th electrical ($E$) component of Green's function induced 
by $j$th component of electrical ($E$) source; $G_{ij}^{H|E}$ is the $i$th 
magnetic ($H$) component of Green's function induced by $j$th component of 
electrical ($E$) source. The representation theorem gives
\begin{equation}\label{eq:green1}
  E_i =  G_{ij}^{E|E} J_j, H_i = G_{ij}^{H|E} J_j.
\end{equation}



We can do the same assuming only a magnetic source $M_j=\delta(x-x_s)e_j$: 
$G_{ij}^{E|H}$ is the $i$th electrical ($E$)  component of Green's function 
induced by $j$th component of magnetic ($H$) source; $G_{ij}^{H|H}$ is the 
$i$th magnetic ($H$) component of Green's function induced by $j$th component of 
magnetic ($H$) source.
\begin{equation}\label{eq:HH|E|H}
  \begin{cases}
    \nabla \times G_{ij}^{E|H} -\mathrm{i}\omega\mu  G_{ij}^{H|H} &= \delta(x-x_s)e_j \\
    \nabla \times G_{ij}^{H|H} -\sigma  G_{ij}^{E|H}  &=0
  \end{cases},
\end{equation}
which defines another two Green's function $G_{ij}^{E|H}$ and $G_{ij}^{H|H}$. Similar to equation \eqref{eq:green1}, the representation theorem gives
\begin{equation}
  E_i =  G_{ij}^{E|H} M_j, H_i = G_{ij}^{H|H} M_j.
\end{equation}
The total electrical  and magnetic fields in the coupled system is then the 
superposition of two contributions:
\begin{equation}
  E_i =\sum_j G_{ij}^{E|E} J_j + G_{ij}^{E|H} M_j, \quad 
  H_i =\sum_j G_{ij}^{H|E} J_j + G_{ij}^{H|H} M_j.
\end{equation}
It is shown that the reciprocity for EM system holds in the following form
\begin{equation}\label{eq:reciprocity}
  \begin{cases}
  G_{ij}^{E|E}(x_s|x_r) =  G_{ji}^{E|E}(x_r|x_s),\\
  G_{ij}^{H|H}(x_s|x_r) =  G_{ji}^{H|H}(x_r|x_s),\\
  G_{ij}^{H|E}(x_s|x_r) = - G_{ji}^{E|H}(x_r|x_s).
  \end{cases}
\end{equation}

Without magnetic source, we have 
\begin{equation}
  \underbrace{\begin{bmatrix}
    E_x\\
    E_y\\
    E_z
  \end{bmatrix}}_E =\underbrace{\begin{bmatrix}
      G_{xx}^{E|E} &G_{xy}^{E|E} & G_{xz}^{E|E} \\
      G_{yx}^{E|E} &G_{yy}^{E|E} & G_{yz}^{E|E} \\
      G_{zx}^{E|E} &G_{zy}^{E|E} & G_{zz}^{E|E} \\
  \end{bmatrix}}_{G^{E|E}}\underbrace{\begin{bmatrix}
    J_x\\
    J_y\\
    J_z
  \end{bmatrix}}_{J_s} ,
  \underbrace{\begin{bmatrix}
    H_x\\
    H_y\\
    H_z
  \end{bmatrix}}_H =\underbrace{\begin{bmatrix}
      G_{xx}^{H|E} &G_{xy}^{H|E} & G_{xz}^{H|E} \\
      G_{yx}^{H|E} &G_{yy}^{H|E} & G_{yz}^{H|E} \\
      G_{zx}^{H|E} &G_{zy}^{H|E} & G_{zz}^{H|E} \\
  \end{bmatrix}}_{G^{H|E}}\underbrace{\begin{bmatrix}
    J_x\\
    J_y\\
    J_z
  \end{bmatrix}}_{J_s}.
\end{equation}
If $J_s|_{x=x_s}=(1,0,0)^\mathrm{T}$, we have the 1st column of the matrix $G^{E|E}$ and $G^{H|E}$ extracted from vector fields $E$ and $H$
\begin{subequations}
  \begin{align}
  \begin{bmatrix}
    E_x(x_r)\\
    E_y(x_r)\\
    E_z(x_r)
  \end{bmatrix} = &\begin{bmatrix}
    G_{xx}^{E|E}(x_r|x_s)\\
    G_{yx}^{E|E}(x_r|x_s)\\
    G_{zx}^{E|E}(x_r|x_s)\\
  \end{bmatrix}=\begin{bmatrix}
  G_{xx}^{E|E}(x_s|x_r)\\
  G_{xy}^{E|E}(x_s|x_r)\\
  G_{xz}^{E|E}(x_s|x_r)\\
  \end{bmatrix},\\
  \begin{bmatrix}
    H_x(x_r)\\
    H_y(x_r)\\
    H_z(x_r)
  \end{bmatrix} = &\begin{bmatrix}
    G_{xx}^{H|E}(x_r|x_s)\\
    G_{yx}^{H|E}(x_r|x_s)\\
    G_{zx}^{H|E}(x_r|x_s)\\
  \end{bmatrix}=-\begin{bmatrix}
  G_{xx}^{E|H}(x_s|x_r)\\
  G_{xy}^{E|H}(x_s|x_r)\\
  G_{xz}^{E|H}(x_s|x_r)\\
  \end{bmatrix},
  \end{align}
\end{subequations}
where the last equality comes from the reciprocity in \eqref{eq:reciprocity}. It implies that by switching the source and receiver position, we can reproduce the $E_x$, $E_y$ and $E_z$ response from the $E_x$-channel of the receiver at source position by repeating the modeling using the electrical sources at receiver location, i.e., $J_s|_{x=x_r}=(1,0,0)^\mathrm{T}$, $J_s|_{x=x_r}=(0,1,0)^\mathrm{T}$ and $J_s|_{x=x_r}=(0,0,1)^\mathrm{T}$. Similarly, we should reproduce  $-H_x$, $-H_y$ and $-H_z$ response from  the $E_x$-channel of the receiver at source position by repeating the modeling using the magnetic sources at receiver location, i.e., $M_s|_{x=x_r}=(1,0,0)^\mathrm{T}$, $M_s|_{x=x_r}=(0,1,0)^\mathrm{T}$ and $M_s|_{x=x_r}=(0,0,1)^\mathrm{T}$.


If $J_s|_{x=x_s}=(0,1,0)^\mathrm{T}$, we have the 2nd column of the matrix $G^{E|E}$ and $G^{H|E}$ extracted from vector fields $E$ and $H$
\begin{subequations}
  \begin{align}
  \begin{bmatrix}
    E_x(x_r)\\
    E_y(x_r)\\
    E_z(x_r)
  \end{bmatrix} = &\begin{bmatrix}
    G_{xy}^{E|E}(x_r|x_s)\\
    G_{yy}^{E|E}(x_r|x_s)\\
    G_{zy}^{E|E}(x_r|x_s)\\
  \end{bmatrix}=\begin{bmatrix}
  G_{yx}^{E|E}(x_s|x_r)\\
  G_{yy}^{E|E}(x_s|x_r)\\
  G_{yz}^{E|E}(x_s|x_r)\\
  \end{bmatrix},\\
  \begin{bmatrix}
    H_x(x_r)\\
    H_y(x_r)\\
    H_z(x_r)
  \end{bmatrix} =& \begin{bmatrix}
    G_{xy}^{H|E}(x_r|x_s)\\
    G_{yy}^{H|E}(x_r|x_s)\\
    G_{zy}^{H|E}(x_r|x_s)\\
  \end{bmatrix}=-\begin{bmatrix}
  G_{yx}^{E|H}(x_s|x_r)\\
  G_{yy}^{E|H}(x_s|x_r)\\
  G_{yz}^{E|H}(x_s|x_r)\\
  \end{bmatrix}.
  \end{align}
\end{subequations}
By switching the source and receiver position, we obtain the $E_x$, $E_y$ and $E_z$ response from the $E_y$-channel of the receiver at source position by repeating the modeling placing the sources at receiver location, i.e., $J_s|_{x=x_r}=(1,0,0)^\mathrm{T}$, $J_s|_{x=x_r}=(0,1,0)^\mathrm{T}$ and $J_s|_{x=x_r}=(0,0,1)^\mathrm{T}$. Also, we obtain  $-H_x$, $-H_y$ and $-H_z$ response from  the $E_y$-channel of the receiver at source position by repeating the modeling placing the magnetic sources at receiver location, i.e., $M_s|_{x=x_r}=(1,0,0)^\mathrm{T}$, $M_s|_{x=x_r}=(0,1,0)^\mathrm{T}$ and $M_s|_{x=x_r}=(0,0,1)^\mathrm{T}$.



\end{document}
